\documentclass{beamer}

% Step one: Pick a theme!
\usetheme{Ilmenau}
% Available themes are:
% AnnArbor, Antibes, Bergen, Berkeley, Berlin, Boadilla, boxes,
% CambridgeUS, Copenhagen, Darmstadt, default, Dresden, Frankfurt,
% Goettingen, Hannover, Ilmenau, JuanLesPins, Luebeck, Madrid, Malmoe,
% Marburg, Montpellier, PaloAlto, Pittsburgh, Rochester, Singapore,
% Szeged, Warsaw

% Want to customize the color?  Use the following command.  Beaver
% matches pretty well with Huntingdon's colors  :)
\usecolortheme{beaver}
% albatross, beaver, beetle, crane, default, dolphin, dove, fly, lily,
% orchid, rose, seagull, seahorse, sidebartab, structure, whale,
% wolverine

% Use the listings package for including code (in this case, LaTeX).
% If you don't include any code, then you needn't worry about it.  If
% you do, you can either look up the listings package or simply use
% the verbatim package.  Either way, come see me and I'll show you
% how.  The several lines following includes the listing packages and
% defines some nice defaults (lifted from
% http://en.wikibooks.org/wiki/LaTeX/Source_Code_Listings)
\usepackage{listings}
\definecolor{mygreen}{rgb}{0,0.6,0}
\definecolor{mygray}{rgb}{0.5,0.5,0.5}
\definecolor{mymauve}{rgb}{0.58,0,0.82}
\lstset{backgroundcolor=\color{white}, basicstyle=\footnotesize,
  breakatwhitespace=false, breaklines=true, captionpos=b,
  commentstyle=\color{mygreen}, deletekeywords={},
  escapeinside={@}{@},
  extendedchars=true, frame=single,
  keywordstyle=\color{blue}, language={[LaTeX]TeX},
  morekeywords={\mathbb,\mathcal,\mathfrak}, numbers=left, numbersep=5pt,
  numberstyle=\tiny\color{mygray}, rulecolor=\color{black},
  showspaces=false, showstringspaces=false, showtabs=false,
  stringstyle=\color{mymauve}, tabsize=2, title=\lstname}

% The following lets you include URLs in a \url{...}.
% \usepackage{url}

\begin{document}
\title[Beamer]{A Beamer Example} 
\subtitle{With code!}
\author[C. Curry]{Clinton Curry}\institute{Huntingdon
  College}

\date[2013/04/07]{Capstone Class\\April 7, 2013}

\begin{frame}
\titlepage
\end{frame}

\begin{frame}{Outline}
  \tableofcontents
\end{frame}

\section{How to get started}

\begin{frame}[fragile]{How to begin}
  Every Beamer presentation should start something like 
\begin{lstlisting}
\documentclass{beamer} 
...  % preamble goes here
\begin{document}
\title{Your Title}
\author{Your Name}
\begin{frame}
\titlepage
\end{@@frame}
...  % Your presentation goes here
\end{document}
\end{lstlisting}
\end{frame}

\begin{frame}[fragile]{Making Frames}
  A \textbf{frame} is a unit of display -- roughly, one per slide.
  You use the \verb|frame| environment to create them.  Remember to
  give your frames a title!

  Put a \verb|frame| environment between your \verb|\begin{document}|
    and \verb|\end{document}|.
\begin{lstlisting}[language={[LaTeX]TeX}]
\begin{frame}{Making Frames}
  A \textbf{frame} is a unit of display -- roughly, 
  one per slide. You use the \verb|frame| environment 
  to create them.  Remember to give your frames a
  title!
\end{@@frame}@%Had to do this, otherwise I got an error...@
\end{lstlisting}
\end{frame}
\begin{frame}[fragile]{Table of Contents}
  Right after your title slide, it is useful to include a frame with
  the table of contents.  (You'll have to compile your document a
  couple of times after updating a section name for the table of
  contents to change.)

\begin{lstlisting}
\begin{frame}{Outline}
\tableofcontents
\end{@@frame}
\end{lstlisting}
\end{frame}
\begin{frame}[fragile]{Sectioning}
  You can use the \verb|\section| and \verb|\subsection| commands as
  usual.  These make the sections in your talk, and show up on your
  table of contents slide.
\end{frame}
\section{Math Commands}
\begin{frame}[fragile]{Math!}
  You can do the same kinds of mathematically wondrous things that you
  could do before.  These include:
  \begin{itemize}
  \item Inline math expressions: $e^{100} \approx 2.69 \times
    10^{43}$.
    \begin{lstlisting}
$e^{100} \approx 2.69 \times 10^{43}$
    \end{lstlisting}
    \item Displayed math expressions:
      \[ \int_{-\infty}^\infty e^{-x^2}\,dx = \sqrt{\pi} \]
\begin{lstlisting}
\[ 
  \int_{-\infty}^\infty e^{-x^2}\,dx = \sqrt{\pi}
\]
\end{lstlisting}
  \end{itemize}
\end{frame}
\begin{frame}[fragile]{\AmS-\LaTeX commands}
  You also have all \verb|amsmath|,\verb|amsthm|, \verb|amsfonts|, and
  \verb|amssymb| commands available to you.  See the documentation:
  
  \begin{itemize}
  \item \href{ftp://ftp.ams.org/pub/tex/doc/amsmath/amsldoc.pdf}{amsldoc.pdf}
  \item \href{ftp://ftp.ams.org/pub/tex/doc/amscls/amsthdoc.pdf}{amsthdoc.pdf}
  \end{itemize}
\end{frame}
\begin{frame}{\texttt{amsthm} commands}
  \begin{lemma}
    \LaTeX{} is cool.
  \end{lemma}
  \begin{proof}
    Apparent from observation.
  \end{proof}
  \begin{theorem}
    Beamer is cool!
  \end{theorem}
  \begin{proof}[Proof by name-calling]
    Anyone who disagrees is a dweeb.
  \end{proof}
\end{frame}
\begin{frame}[fragile]{\texttt{amsthm} commands}
The \verb|lemma|, \verb|theorem|, \verb|corollary|, and \verb|proof|
commands are automatically defined.
\begin{lstlisting}
\begin{lemma}
  \LaTeX{} is cool.
\end{lemma}
\begin{proof}
  Apparent from observation.
\end{proof}
\begin{theorem}
  Beamer is cool!
\end{theorem}
\begin{proof}[Proof by name-calling]
  Anyone who disagrees is a dweeb.
\end{proof}
\end{lstlisting}
\end{frame}

\begin{frame}[fragile]{amsfonts}
  You have \verb|\mathbb|, \verb|\mathrm|, \verb|\mathbf|,
  \verb|\mathcal|, and \verb|\mathfrac|, as usual.
  \begin{itemize}
  \item Normal math: $ABCDEFGHIJKLMNOPQRSTUVWXYZ$
  \item \verb|\mathbb|: $\mathbb{ABCDEFGHIJKLMNOPQRSTUVWXYZ}$
  \item \verb|\mathbf|: $\mathbf{ABCDEFGHIJKLMNOPQRSTUVWXYZ}$
  \item \verb|\mathcal|: $\mathcal{ABCDEFGHIJKLMNOPQRSTUVWXYZ}$
  \item \verb|\mathfrak|: $\mathfrak{ABCDEFGHIJKLMNOPQRSTUVWXYZ}$
  \item \verb|\mathrm|: $\mathrm{ABCDEFGHIJKLMNOPQRSTUVWXYZ}$
  \end{itemize}
\end{frame}
\begin{frame}[fragile]{amsfonts}
\begin{lstlisting}
\begin{itemize}
\item Normal math: $ABCDEFGHIJKLMNOPQRSTUVWXYZ$
\item \verb|\mathbb|: $\mathbb{ABCDEFGHIJKLMNOPQRSTUVWXYZ}$
\item \verb|\mathbf|: $\mathbf{ABCDEFGHIJKLMNOPQRSTUVWXYZ}$
\item \verb|\mathcal|: $\mathcal{ABCDEFGHIJKLMNOPQRSTUVWXYZ}$
\item \verb|\mathfrak|: $\mathfrak{ABCDEFGHIJKLMNOPQRSTUVWXYZ}$
\item \verb|\mathrm|: $\mathrm{ABCDEFGHIJKLMNOPQRSTUVWXYZ}$
\end{itemize}
\end{lstlisting}
\end{frame}
\begin{frame}[fragile]{amssymb}
  You can use all of the many special-purpose symbols defined by
  \texttt{amssymb}, which you can find in the
  \href{http://mirrors.ctan.org/info/symbols/comprehensive/symbols-letter.pdf}{\color{blue}Comprehensive
    \LaTeX{} Symbol List}.
\end{frame}
\begin{frame}[fragile]{amsmath}
  You can use the many excellent facilities of the \verb|amsmath|
  package.  This includes
  \begin{itemize}
  \item the \verb|\text| command, to include text in math mode;
  \item the \verb|align| environment, to line up equations like
    \begin{align*}
      (x+1)(x-1) &= x(x-1) + 1(x-1) & \text{distribute}\\
      &= x^2 - x + x - 1 & \text{distribute}\\
      &= x^2 - 1 & \text{combine}
    \end{align*} and
  \item matrices like
    \[
    \begin{pmatrix}
      1 & 3 & 5\\
      4 & 7 & -1
    \end{pmatrix}
    \begin{bmatrix}
      9 \\ 5 \\3
    \end{bmatrix} = 
    \begin{bmatrix}
      (9)(1)+(5)(3)+(3)(5)\\
      (9)(4)+(5)(7)+(3)(-1)
    \end{bmatrix}
    =
    \begin{bmatrix}
      39\\68
    \end{bmatrix}
    \]
  \end{itemize}
\end{frame}
\begin{frame}[fragile]{amsmath}
\begin{lstlisting}
% Alignment occurs on ampersands
\begin{align*}
  (x+1)(x-1) &= x(x-1) + 1(x-1) & \text{distribute}\\
      &= x^2 - x + x - 1 & \text{distribute}\\
      &= x^2 - 1 & \text{combine}
\end{align*}

% pmatrix uses parentheses, bmatrix uses
% square brackets
\[
\begin{pmatrix}
  1 & 3 & 5\\
  4 & 7 & -1
\end{pmatrix}
\]
\end{lstlisting}
\end{frame}


\begin{frame}[fragile]{Pausing}
  Using the \verb|\pause| command, you can hold off on showing
  something until appropriate.
  \pause
  \begin{center}
    \Huge Like this!
  \end{center}
  \pause
\begin{lstlisting}
Using the \verb|\pause| command, you can hold off on showing something until appropriate.
\pause
\begin{center}
\Huge Like this!
\end{center}
\end{lstlisting}
\end{frame}
\section{Various pretty things}
\begin{frame}[fragile]{Pictures}
  Pictures can be included with the \verb|\includegraphics| command;
  this can be used for PNG, JPEG, or PDF files.
  \begin{center}
    \includegraphics[height=2in]{TFZsuperellipse-crop.pdf}
  \end{center}
\end{frame}
\begin{frame}[plain]
  \begin{center}
    \includegraphics[height=3.5in]{TFZsuperellipse-crop.pdf}
  \end{center}
\end{frame}

\begin{frame}[fragile]{Pictures}
\begin{lstlisting}
\begin{center}
  \includegraphics[height=2in]{filename.pdf}
\end{center}
\end{lstlisting}
The entire slide is only about 3.5 inches tall.  You can get the full
height by using the [plain] option on the frame:
\begin{lstlisting}
\begin{frame}[plain]{Title here}
\begin{center}
  \includegraphics[height=3.5in]{filename.pdf}
\end{center}
\end{@@frame}
\end{lstlisting}
\end{frame}

\begin{frame}{Further Information}
  Another tutorial is available from
  \href{http://www.uncg.edu/cmp/reu/presentations/Charles\%20Batts\%20-\%20Beamer\%20Tutorial.pdf}{\color{blue}here}
  (credit to Charles Batts).
  
  \bigskip

  If you really want to drink from the firehose, check out
    \href{http://mirrors.ctan.org/macros/latex/contrib/beamer/doc/beameruserguide.pdf}%
    {\color{blue}the beamer user guide}.
\end{frame}


\end{document}
